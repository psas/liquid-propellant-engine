


\maketitle
\end{Verbatim}

\subsection{Headings}

AIAA Technical Conferences style defines 3 levels of section
headings:
\begin{description}
 \item Level 1 heading (\verb"\section"): bold, larger font, centered,
       and numbered with Roman numerals.
 \item Level 2 heading (\verb"\subsection"): bold, flush left, and
       numbered with capital letters.  No opening paragraph indentation.
 \item Level 3 heading (\verb"\subsubsection"): italic, flush left, and
       numbered with Arabic numbers (e.g., 1, 2, 3).  No opening
       paragraph indentation.
\end{description}

\subsection{Abstract}

The abstract should appear at the beginning of your paper.
It should be one paragraph long (not an introduction) and complete in
itself (no reference numbers).
It should indicate subjects dealt with in the paper and state the
objectives of the investigation.
Newly observed facts and conclusions of the experiment or argument
discussed in the paper must be stated in summary form; readers should
not have to read the paper to understand the abstract.

\subsection{Footnotes and References}

Footnotes, where they appear, should be placed above the 1-inch margin at
the bottom of the page.
To insert footnotes, use \verb|\footnote{}| as normal.
Footnotes are formatted automatically in \verb|aiaa.cls|, but if
another medium is used, they should appear in as superscript lower case
letters.
When adding notes to tables, e.g., as accommodated by the
\pkg{threeparttable} package, the symbols are down by hand and should be
in the sequence, *, $\dagger$, $\ddagger$, $\mathsection$,
$\mathparagraph$, $\|$, **, $\dagger\dagger$, and so forth.
This sequence should beginning anew with each table.

List and number all bibliographical references at the end of the paper.
Corresponding superscript numbers are used to cite references in the
text using the \verb|\cite{}| command,\cite{sutton:85ar} unless the
citation is an integral part of the sentence (e.g., ``It is shown in
Ref.~\citen{miner:75ncr} that\ldots[.]'') 
or follows a mathematical expression: ``$A2 + B = C$
(Ref.~\citen{wirin:90cp}).''
For this case, the command, \verb|\citen{}| is used.
\cite{*}% to load all refences for demonstration purposes which follow
Multiple citations are sorted and punctuated automatically through the
\cls{aiaa-tc}'s use of the \pkg{overcite}. 
Separate reference numbers are shown with
commas\cite{sutton:85ar,miner:75ncr,turner:64a} and ranges are 
separated by an en-dash.\cite{sutton:85ar,miner:75ncr,wirin:90cp}
Reference citations in the text should be in numerical order, which is
assured by using \textsc{Bib}\TeX.

In the reference list, give all authors' names; do not use ``et al.''
unless there are six authors or more.
Papers that have not been published should be cited as ``unpublished'';
papers that have been submitted or accepted for publication should be
cited as ``submitted for publication.''
Private communications and personal Web sites should appear as footnotes
rather than in the reference list.

References should be cited according to the standard publication
reference style.
(For examples, see the ``References'' section of this template.)
This is facilitated by the Bib\TeX\ database and a style file, \file{aiaa.bst}.
As a rule, all words are capitalized except for articles, conjunctions,
and prepositions of four letters or fewer.
Names and locations of publishers should be listed; month and year
should be included for reports and papers.
For papers published in translation journals, please give the English
citation first, followed by the original foreign language citation.

\subsection{Images, Figures, and Tables}

All artwork, captions, figures, graphs, and tables will be reproduced
exactly as submitted.
Be sure to position any figures, tables, graphs, or pictures as you want
them printed.
AIAA is not responsible for incorporating your figures, tables,
etc.%
\footnote{Company logos and identification numbers are not permitted on
  your illustrations.}

% an experiment with wrapfig -- read caveats documented in wrapfig.sty
\begin{wrapfigure}{R}{0.5\linewidth}
 \includegraphics{figure_magnet}
 \caption{Magnetization as a function of applied field, which has
   borders so thick that they overwhelm the data and for some reason the
   ordinate label is rotated 90 degrees to make it difficult to
   read. This figure also demonstrates the dangers of using a bitmap
   as opposed to a vector image.}
 \label{f:magnetic_field}
\end{wrapfigure}

Place figure captions below all figures; place table titles above the
tables.%
\footnote{Please do not include captions as part of the figure image
itself.}
If your figure has multiple parts, include the labels ``a),'' ``b),''
and so on, below each subfigure and above the figure caption.
This can be accomplished with the \pkg{subfigure} package.
Please verify that the figures and tables you mention in the text
actually exist by using \LaTeX's \verb|\label| and \verb|\ref| mechanisms
and verifying that there are no undefined references during typesetting.

When citing a figure in the text, use ``figure'' as ``Fig., often used
to refer to graphics, is an ugly abbreviation and is not worth the two
spaces saved.''\cite{tufte:83bk}
Do not abbreviate ``table'' either.
Each different type of illustration (i.e., figures, tables, or
images) should be numbered sequentially with relation to other
illustrations of the same type.

Figure axis labels are often a source of confusion.
Use words rather than symbols.
As in the example to the right, write the quantity ``Magnetization''
rather than just ``M.''
Do not enclose units in parenthesis, but rather separate them from the
preceding text by commas.
Do not label axes only with units.
As in figure~\ref{f:magnetic_field}, for example, write ``Magnetization,
A/m'' or ``Magnetization, A m-1,'' not just ``A/m.''
Do not label axes with a ratio of quantities and units.
For example, write ``Temperature, K,'' not ``Temperature/K.''

Multipliers can be especially confusing.
Write ``Magnetization, kA/m'' or ``Magnetization, 103 A/m.''
Do not write ``Magnetization (A/m) $\times$ 1000'' because the reader
would not then know whether the top axis label in
figure~\ref{f:magnetic_field} meant 16000~A/m or 0.016~A/m.
Figure labels must be legible, approximately 8--12 point type.
Use of the \pkg{psfrag} package can facilitate this, but it makes
generating a PDF version of your paper difficult.

Figures and tables are referred to as `Floats' in \LaTeX, reflecting
their floating nature.
They are typically numbered whether or not they have a caption and they
are floated to the first available position near the first reference to
that figure/table within the text.
This is accomplished by placing the figure/table environment after
their first reference.
The environment can even be placed mid-sentence as long as \LaTeX's
comment character, \%, is used properly to prevent a paragraph break or
extra spaces.
By default, floats are placed at the top of the first available page,
the bottom of the page, or one a page consisting entirely of floats.

An example of coding a standard \verb|{figure}| is given below:
\begin{Verbatim}
 \begin{figure}
   \includegraphics{magetic_field}
   \caption{Magnetization as a function of applied field, which has
     borders so thick that they overwhelm from the data.}
   \label{f:magnetic_field}
 \end{figure}
\end{Verbatim}

\begin{table}
 \begin{center}
  \begin{threeparttable}
   \caption{This is an example of a \pkg{threeparttable} which uses the
     \pkg{dcolumn} package to allow for columns to be aligned on decimal
     points.}
   \label{t:threepart}
   \begin{tabular}{lcdd}
    First head\tnote{*}  &
    Second head &
    \multicolumn{1}{c}{Third head} &
    \multicolumn{1}{c}{$V_M(r)$} \\\hline
    center & doctor &  0.2  & 10.55 \\  
    tab    & dentist &  0.15 & 33.12 \\ 
    worse  & man\tnote{\ensuremath{\dagger}} & 10.58 & 45.10 \\ 
    better & home & 43.9  & 12.34 \\
   \end{tabular}
   \begin{tablenotes}
    \item[*] This is a table footnote, which to span multiple lines, has
      been greatly extended in length contrary to reason.
    \item[\ensuremath{\dagger}] A much shorter table footnote.
   \end{tablenotes}
  \end{threeparttable}
 \end{center}
\end{table}
Tables with footnotes, such as table~\ref{t:threepart},
can be coded using the \pkg{threeparttable} package as follows:
\begin{Verbatim}
 \begin{table}
  \begin{center}
   \begin{threeparttable}
    \caption{This is an example of table caption.}
    \label{t:wordtable}
    \begin{tabular}{ccdd}
     First head\tnote{a}  &
     Second head\tnote{b} &
     \multicolumn{1}{c}{Third head} &
     \multicolumn{1}{c}{$V_M(r)$} \\\hline
     Left & Word entries &  0.2  & 10.55 \\  
     Left & Word entries &  0.15 & 33.12 \\ 
     Left & Word entries & 10.58 & 45.10 \\ 
     Left & Word entries & 43.9  & 12.34 \\
    \end{tabular}
    \begin{tablenotes}
     \item[a] This is a table footnote, which to span multiple lines, has
              been greatly extended in length contrary to reason.
     \item[b] A much shorter table footnote.
    \end{tablenotes}
   \end{threeparttable}
  \end{center}
 \end{table}
\end{Verbatim}

\subsection{Equations, Numbers, Symbols, and Abbreviations}

Equations are centered and numbered consecutively, with equation
numbers in parentheses flush right, as in Eq.~\ref{e:function}.
\begin{equation}
 \label{e:function}
 \int_{0}^{r_{2}} F(r,\varphi)\,dr\,d\varphi =
    \left[ \sigma r_{2}/(2\mu_{0}) \right] \cdot
    \int_{0}^{\infty} \exp(-\rho|z_{j}-z_{i}|) \lambda^{-1} 
\end{equation}
Equation~\ref{e:function} is coded as below:
\begin{Verbatim}
 \begin{equation}
  \label{e:function}
  \int_{0}^{r_{2}} F(r,\varphi)\,dr\,d\varphi =
     \left[ \sigma r_{2}/(2\mu_{0}) \right] \cdot
     \int_{0}^{\infty} \exp(-\rho|z_{j}-z_{i}|) \lambda^{-1}
 \end{equation}
\end{Verbatim}
A sample equation is included below, formatted using the preceding
instructions.
To make your equation more compact, you can use the solidus ($/$), the
exp function, or appropriate exponents.
Use parentheses to avoid ambiguities in denominators.
\begin{equation}
  \label{e:displace}
  \mathbf{J}_i\cdot\Delta\underline{x}_{i+1}=-\underline{f}_i
\end{equation}
\nomenclature{$J$}{Jacobian Matrix}
\nomenclature[b$i$]{$i$}{Variable number}
\nomenclature[g$\Delta$]{$\Delta x$}{Variable displacement vector}
\nomenclature{$f$}{Residual value vector}
\nomenclature{$x$}{Variable value vector}
\begin{equation}
  \label{e:newton}
  F=m\alpha
\end{equation}
\nomenclature{$F$}{Force, N}
\nomenclature{$m$}{Mass, kg}
\nomenclature[g$\alpha$]{$\alpha$}{Acceleration, m/s\textsuperscript{2}}

The \verb|fleqnarray| environment is used to display a sequence of
equations or inequalities. It is very much like a three-column
\verb|array| environment, with consecutive rows separated by
\verb|\\| and consecutive items within a row separated by an \&.
\begin{eqnarray}
&&\sum_{i\in H} p^0\cdot w^{0i}=\sum_i M^i (p^0) =
  \sum_i\left[p^0\cdot r^i + \sum_j \alpha^{ij} (p^0\cdot
  y^{0j})\right]\nonumber\\
&&{\qquad}= p^0 \cdot \sum_i r^i + p^0 \cdot \sum_i
\sum_i \alpha^{ij}
\end{eqnarray}
An equation number is placed on every line unless that line has a
\verb|\nonumber| command.

Be sure that the symbols in your equation are defined before the
equation appears, or immediately following.
In this case the \pkg{nomencl} package is used to automatically generate
the nomenclature section.

Italicize symbols ($T$ might refer to temperature, but T is the unit
Tesla).
Refer to ``Eq. (1),'' not ``(1)'' or ``equation (1)'' except at the
beginning of a sentence: ``Equation (1) is phat.''
Equations can be labeled other than ``Eq.'' should they represent
inequalities, matrices, or boundary conditions.
If what is represented is really more than one equation, the
abbreviation ``Eqs.'' can be used.

Define abbreviations and acronyms the first time they are used in the
text, even after they have already been defined in the abstract.
Very common abbreviations such as AIAA, SI, ac, and dc do not have to be
defined.
Abbreviations that incorporate periods should not have spaces: write
``P.R.,'' not ``P. R.''
Delete periods between initials if the abbreviation has three or more
initials; e.g., U.N. but ESA.\@
Do not use abbreviations in the title unless they are unavoidable (for
instance, ``AIAA'' in the title of this article).

\subsection{General Grammar and Preferred Usage}

Use only one space after periods or colons. Hyphenate complex
modifiers: ``zero-field-cooled magnetization.''
Avoid dangling participles, such as, ``Using Eq. (1), the potential was
calculated.''
[It is not clear who or what used Eq. (1).]
Write instead ``The potential was calculated using Eq. (1),'' or ``Using
Eq. (1), we calculated the potential.''

Use a zero before decimal points: ``0.25,'' not ``.25.''
Use ``cm$^3$,'' not ``cc.''
Indicate sample dimensions as ``0.1 cm $\times$ 0.2 cm,'' not ``0.1
$\times$ 0.2~cm$^2$.''
The preferred abbreviation for ``seconds'' is ``s,'' not ``sec.''
Do not mix complete spellings and abbreviations of units: use
``Wb/m$^2$'' or ``webers per square meter,'' not ``webers/m$^2$.''
When expressing a range of values, write ``7 to 9'' or ``7--9,'' not
``7~9.''

A parenthetical statement at the end of a sentence is punctuated
outside of the closing parenthesis (like this).
(A parenthetical sentence is punctuated within parenthesis.)
In American English, periods and commas are placed within quotation
marks, like ``this period.''
Other punctuation is ``outside''! Avoid contractions; for example, write
``do not'' instead of ``don't.''
The serial comma is preferred: ``A, B, and C'' instead of ``A, B and C.''

If you wish, you may write in the first person singular or plural and
use the active voice (``I observed that\ldots'' or ``We observed that\ldots''
instead of ``It was observed thatÖ'').
Remember to check spelling.
If your native language is not English, please ask a native
English-speaking colleague to proofread your paper.

The word ``data'' is plural, not singular (i.e., ``data are,'' not ``data
is'').
The subscript for the permeability of vacuum $\mu_0$ is zero, not a
lowercase letter ``oo.''
The term for residual magnetization is ``remanence''; the adjective is
``remanent''; do not write ``remnance'' or ``remnant.''
The word ``micrometer'' is preferred over ``micron'' when spelling out
this unit of measure.
A graph within a graph is an ``inset,'' not an ``insert.''
The word ``alternatively'' is preferred to the word ``alternately''
(unless you really mean something that alternates).
Use the word ``whereas'' instead of ``while'' (unless you are referring
to simultaneous events).
Do not use the word ``essentially'' to mean ``approximately'' or
``effectively.''
Do not use the word ``issue'' as a euphemism for ``problem.''
When compositions are not specified, separate chemical symbols by
en-dashes; for example, ``NiMn'' indicates the intermetallic compound
Ni$_{0.5}$Mn$_{0.5}$ whereas ``Ni--Mn'' indicates an alloy of some
composition Ni$_{x}$Mn$_{1-x}$.

Be aware of the different meanings of the homophones ``affect'' (usually
a verb) and ``effect'' (usually a noun), ``complement'' and
``compliment,'' ``discreet'' and ``discrete,'' ``principal'' (e.g.,
``principal investigator'') and ``principle'' (e.g., ``principle of
measurement''). Do not confuse ``imply'' and ``infer.''

Prefixes such as ``non,'' ``sub,'' ``micro,'' ``multi,'' and ``ultra''
are not independent words; they should be joined to the words they
modify, usually without a hyphen.
There is no period after the ``et'' in the abbreviation ``et al.''
The abbreviation ``i.e.,'' means ``that is,'' and the abbreviation
``e.g.,'' means ``for example'' (these abbreviations are not
italicized).
An excellent source of more detailed style and formatting instructions
can be found in the AIAA style guide, {\it AIAA Style} (available from AIAA
upon request).

\section{Conclusion}

A conclusion section is not required, though it is preferred.
Although a conclusion may review the main points of the paper, do not
replicate the abstract as the conclusion.
A conclusion might elaborate on the importance of the work or suggest
applications and extensions.%
\footnote{The conclusion section is the last section of
  the paper that should be numbered.
  The appendix (if present), acknowledgment, and references should be
  listed without numbers.}

\section*{Appendix}

An appendix, if needed, should appear before the acknowledgments.
Use the 'starred' version of the \verb|\section| commands to avoid
section numbering.

\section*{Acknowledgments}

The preferred spelling of the word ``acknowledgment'' in American
English is without the ``e'' after the ``g.''
Avoid expressions such as ``One of us (S.B.A.) would like to thank\ldots[.]''
Instead, write ``F. A. Author thanks\ldots[.]''
Sponsor and financial support acknowledgments are also to be listed in
the acknowledgments section.

% produces the bibliography section when processed by BibTeX
\bibliography{bibtex_database}
\bibliographystyle{aiaa}

